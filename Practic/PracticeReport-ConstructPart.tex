\chapter{Конструкторская часть}

\section {Описание представления объектов сцены}

Аналитический способ представления трехмерных объектов сцены означает, что
для каждой формы придется использовать собственное уравнение (или систему уравнений)

\subsection*{Представление сферы}

Поверхность сферы радиуса r, расположенной в начале координат, аналитечески может быть представлена в виде уравнения
\begin{equation*}
	x^2 + y^2 + z^2 = r^2
\end{equation*}

Следовательно структура данных, описывающя сферу, должна хранить два поля:
\begin{enumerate}
	\item[1)] position : (float, float, float) --- смещение по (x, y, z);
	\item[2)] r : float --- радиус, r > 0.
\end{enumerate}

\subsection*{Представление цилиндра}

Поверхность правильного цилиндра радиуса r и высоты $h = 2\cdot l$, расположенного в начале координат, аналитически может быть представлен в виде системы:
\begin{equation*}
	\begin{cases}
		x^2 + y^2 = r^2\text{, При -l <= z <= l : боковая поверхность}\\
		x^2 + y^2 <= r^2\text{, При |z| = l : основания цилиндра}\\
	\end{cases}
\end{equation*}

Также необходимо учесть, что цилиндр может быть расположен в пространстве под произвольным углом относитьльно осей координат, для чего вводится отдельное поле в структуре данных.

\begin{enumerate}
	\item position : (float, float, float) - смещение по (x, y, z)
	\item rotation : (float, float, float) - поворот относительно (Ox, Oy, Oz)
	\item r : float, r > 0 - радиус
	\item l : float, l > 0 - высота цилиндра
\end{enumerate}

\subsection*{Представление параллелепипеда}

Поверхность прямоугольного параллелепипеда со сторонами, параллельными координатным плоскостям может быть однозначно задана парой вершин параллелепипеда, примыкающих к его диагонали. 

Тогда структура данных, описывающая параллелепипед должна иметь вид.

\begin{enumerate}
	\item p1, p2 : (float, float, float) - вершины
	\item rotation : (float, float, float) - поворот относительно (Ox, Oy, Oz)
\end{enumerate}

\section{Общий алгоритм решения поставленной задачи}

Общий алгоритм решения поставленной задачи включает следующие этапы.

\begin{enumerate}

\item Задать объекты сцены.
\item Задать анимацию.
\item Каждый кадр анимации выполнить:
	\begin{enumerate}
	\item Использовать ключевые кадры для вычисления позиции и вращения объектов.
	\item Применить алгоритм обратной трассировки лучей для синтеза изображения.
	\item Вывести изображение на экран.
	\end{enumerate}

\end{enumerate}

\section{Алгоритм обратной трассировки лучей с использованием модели освещения Фонга}

Ниже описан алгоритм обратной трассировки лучей с использованием параллельных вычислений для модели освещения Фонга. 

%ToDo: проверить, не включатся ли случайно точки (в шаблоне ) после пунктов списка 1 и 2 (в крайнем случае дорисовать гелевой ручкой)

\begin{enumerate}
	\item Разбить проецирующую плоскость на равные участки.
	\item Выделить потоки для обработки участков.
	\item Разбить участок на пиксели.
	\item Для каждого пикселя:
	\begin{enumerate}
		\item Выпустить первичный луч.
		\item Определить пересечения первичного луча со всеми объектами сцены.
		\item Если пересечений нет, покрасить пиксель в цвет фона.
		\item Иначе:
		\begin{enumerate}
				\item Выбрать ближайшую к началу луча точку пересечения.
				\item Из точки пересечения выпустить теневые зонды.
				\item Учесть источники света, до которых дошли теневые зонды.
				\item Покрасить пиксель с соответствующей интенсивностью.
		\end{enumerate}
	\end{enumerate}
\end{enumerate}


\section{Алгоритмы пересечения луча с объектами}

\subsection{Описание алгоритма пересечения луча со сферой}

Поверхность сферы аналитечески может быть представлена в виде следующего уравнения.
\begin{equation*}
	x^2 + y^2 + z^2 = r^2
\end{equation*}
В системе координат сферы рассматривается луч, исходящий из точки $(x0, y0, z0)$ в направлении вектора $(l, m, n)$, где $l^2 + m^2 + n^2 = 1$.

Луч выражается параметрически:

\begin{equation}
	\label{eq-param-ray}
	\begin{cases}
		x = l \cdot t + x0 \\
		y = m \cdot t + y0 \\
		z = n \cdot t + z0,
	\end{cases}
\end{equation}
где $t > 0$.

В уравнении сферы проводится подставновка.
\begin{equation*}
	(l \cdot t + x0)^2 + (m \cdot t + y0)^2 + (n \cdot t + z0)^2 = r^2
\end{equation*}

Получается квадратное уравнение, которое решается через дискриминант.
\begin{equation}
	(l^2 + m^2 + n^2) \cdot t^2 + 2 \cdot (l \cdot x0 + m \cdot y0 + n \cdot z0) \cdot t + (x0^2 + y0^2 + z0^2 - r^2) = 0
\end{equation}

Если квадратное уравнение не имеет решения - луч не пересекает сферу.
Если решение есть, выбирается наименьший положительный t, как параметр, характеризующий ближайшую к началу луча точку пересечения. Отрицательные значения t игнорируются, так как они отвечают за пересечения, находящиеся на одной прямой с лучом, но в противоположном направлении.

\subsection{Описание алгоритма пересечения луча с цилиндром}

Боковая поверхность правильного цилиндра аналитечески может быть представлен в виде следующего уравнения.
\begin{equation*}
		x^2 + y^2 = r^2, \text{ если} -l =< z =< l \\
\end{equation*}
В системе координат цилиндра рассматривается луч, исходящий из точки с координатами $(x0, y0, z0)$ в направлении вектора $(l, m, n)$, где $l^2 + m^2 + n^2 = 1$.
	
Луч выражается параметрически, как в системе \ref{eq-param-ray}.
	
В первую очередь можно проверить луч на пересечение с основаниями цилиндра. Для этого z приравнивается к l и -l, вычисляется значение парамера $t = (z-z0)/n$.

По полученному t вычисляются x, y.

Если полученные точки удовлетворяют $x^2 + y^2 <= r$ То Луч пересекает основания цилиндра. В противном случае ищется пересечение луча с боковой поверхностью.

В уравнении цилиндра проводится подстановка.
\begin{equation*}
(l \cdot t + x0)^2 + (m \cdot t + y0)^2 = r
\end{equation*}

Получается квадратное уравнение, которое решается через дискриминант.
\begin{equation}
	(l^2 + m^2) \cdot t^2 + 2 \cdot (l \cdot x0 + m \cdot y0) \cdot t + (x0^2 + y0^2 - r) = 0
\end{equation}

Если квадратное уравнение не имеет решения, луч не пересекает боковую поверхность цилиндра.

Если решение есть, для полученных значений параметра t вычисляется значение z, которое проверяется на условие --l < z < l.

Если z удовлетворяет условию --- пересечения найдены, выбирается ближайшая к началу луча точка.

\subsection{Описание алгоритма пересечения луча с параллелепипедом}

Прямоугольный параллелепипед со сторонами, параллельными координатным плоскостям, однозначно определяется любыми двумя своими вершинами, примыкающими к одной из диагоналей параллелепипеда.

Например, вершинами $(x1, y1, z1)$, $(x2, y2, z2)$.

В системе координат параллелепипеда рассматривается луч, исходящий из точки $(x0, y0, z0)$ в направлении вектора $(l, m, n)$, где $l^2 + m^2 + n^2 = 1$.

Рассматривается пара плоскостей, параллельных плоскости $yz: x = x1$ и $x = x2$

При l=0 заданный луч параллелен этим плоскостям и, если $x0 < x1$ или $x0 > x2$, то он не пересекает рассматриваемый прямоугольный параллелепипед. Если же l не равно 0, то вычисляются отношения $t1x = (x_{-} - x0)/l$, $t2x = (x_{+} - x0)/l$.

Можно считать, что найденные величины связаны неравенством $t1x < t2x$.

Пусть $tn = t1x$, $tf = t2x$

Считая, что $m > 0$, и рассматривая вторую пару плоскостей, несущих грани заданного параллелепипеда, $y = y1$ и $y = y2$, можно найти вершины величины

$t1y = (y_{-} - y0)/m$, $t2y = (y_{+} - y0)/m$.

Если $t1y > tn$, тогда пусть $tn = t1y$.

Если $t2y < tf$, тогда пусть $tf = t2y$.

При $tn > tf$ или при $tf < 0$ заданный луч проходит мимо прямоугольного параллелепипеда.

Считая $n > 0$, рассматриваем последнюю пару плоскостей $z = z$1 и $z = z2 $, можно найти величины $t1z = (z_{-} - z0)/n$, $t2z = (z_{+} - z0)/n$ затем повторяются предыдущие сравнения.

Если в итоге всех проведенных операций $0 < tn < tf$ или
$0 < tf$, то заданный луч непременно пересечет исходный параллелепипед со сторонами, параллельными координатным осям.

Если луч протыкает прямоугольный параллелепипед (т.е. начальная точка лежит вне параллелепипеда), то расстояние от начала луча до точки его входа в параллелепипед равно $tn$, а до точки выхода луча из параллелепипеда $tf$.

\section{Анимация методом ключевых кадров}

Для анимации методом ключевых кадров необходимо ввести сущность --- кадр. Сущность будет реализована классом KeyFrame. Кадр должен содержать номер кадра анимации, а также смещение и вращение объекта на этот кадр.

Дополнительно будет разработан класс Animation, для хранения списка ключевых кадров, относящихся к одному объекту.

Для отрисовки каждого кадра предварительно будут высчитываться позиция и вращение объекта, как значение между двумя соседними ключевыми кадрами, пропорциональное расстояниям между текущим кадром и ключевыми.