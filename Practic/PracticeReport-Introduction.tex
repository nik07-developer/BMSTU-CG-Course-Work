
\chapter*{Введение}
\addcontentsline{toc}{chapter}{Введение}

В современном мире компьютерная графика используется достаточно широко. Типичная область ее применения --– это кинематография и компьютерные игры.

На сегодняшний день большое внимание уделяется алгоритмам получения реалистичных и динамических изображений (анимации). 
Можно заметить, что чем качественнее мы хотим получить изображение на выходе алгоритма, тем больше времени и памяти потребуется для его синтеза. Это и становится проблемой при создании динамической сцены, так как на каждом временном интервале расчёт визуализации сцены необходимо производить повторно.

Однако не всегда необходимо, чтобы динамическое изображение было реалистичным. Для некоторых задач, например, для демонстрации работы некоторой системы, может быть достаточно обозначить все действующие объекты условно и отобразить эти условные обозначения в программном обеспечении (ПО). Подобной системой может выступать робот Ф-2, который является исследовательской платформой для изучения взаимодействия человека с роботами.

\textbf{Целью} данной работы является разработка ПО с пользовательским интерфейсом, позволяющего получить динамическое изображение трёхмерной сцены с анимированной моделью робота Ф2.

Для достижения поставленной цели требуется решить следующие \textbf{задачи}.

\begin{enumerate}
	\item Описать состав сцены и выбрать форму представления её объектов.
	
	\item Выбрать алгоритм удаления невидимых линий и поверхностей.
	
	\item Выбрать модель освещения.
	
	\item Выбрать способ анимации.

	\item Разработать модель описания сцены и анимации.
	
	\item Разработать алгоритм получения изображения.
	
	\item Реализовать пользовательский интерфейс.
	
	\item Выполнить тестирование и анализ быстродействия программы.
	
\end{enumerate}

\newpage 